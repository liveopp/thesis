\begin{cnabstract}
  子空间聚类是指将一群来自不同低维子空间的数据点按照它们
  所属的空间分开,将同一空间的点聚成一类.本文在稀疏子空间聚类
  方法的基础上,尝试更多利用点的空间性质,提出了基于多任务和
  组稀疏化的新方法.不同于稀疏子空间聚类方法直接将每个点用其他点
  表示,我们先利用点的空间信息,将点分成一个个小组,每个小组作为
  一个整体,希望他们被同时表示,或者同时用来表示别的点.
  这样可以避免只使用一个点带来的不稳定性,更重要的是使随后生成的
  邻接图既没有错误的边,又不至同类连接过于少,从而方便谱聚类达到更好的
  效果.我们通过理论分析,得出新方法在一定条件下能够能使邻接图
  没有错误连接,并且在Hopkins 155数据集和NUS-WIDE数据集上
  与经典SSC,LRR算法做了比较.

  \keywords{子空间聚类\enskip 大数据\enskip 非监督学习\enskip 多任务\enskip 组稀疏}{O29}
\end{cnabstract}

\begin{enabstract}
  In subspace clustering, a group of data points belonging
  to a union of subspaces are assigned membership to their
  respective subspaces. This paper persents a novel method
  that regularizes sparse subspace representation by exploiting the
  structural sharing between tasks and data points via multitasking
  and group sparsity. By doing this, our algorithms are able to be more robust
  without introducing much additional computational cost. The
  theoretical analysis in this paper shows that under certain conditions
  exact clustering performance can be guaranteed. We demonstrate
  the advantage of the framework on Hopkins 155 dataset and NUS-WIDE dataset.

  \enkeywords{Subspace Clustering, Big Data, Unsupervised Learning, Group LASSO,
  Multitasking}{O29}
\end{enabstract}
