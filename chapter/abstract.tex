\begin{cnabstract}
  子空间聚类是指将一群来自不同低维子空间的数据点按照它们
  所属的空间分开,将同一空间的点聚成一类.本文在稀疏子空间聚类的基础上,
  提出了基于多任务和组稀疏的新方法.
  不同于稀疏子空间聚类方法直接将每个点用其他点
  表示,我们先利用点的空间信息,将其分成一个个小组,每个小组作为
  一个整体, 希望他们被同时表示,或者同时用来表示别的点.
  这样既可以减少只使用一个点带来的不稳定,又能使随后生成的
  邻接图相对稠密, 正确连接较多, 从而提升聚类准确度.
  我们通过理论分析,得出新方法在一定条件下能够保证生成的邻接图
  没有错误连接, 并在Hopkins 155和NUS-WIDE数据集上与经典子空间聚类方法比较.
  
  \keywords{子空间聚类\enskip 大数据\enskip 非监督学习\enskip 多任务\enskip 组稀疏}{O24}
\end{cnabstract}

\begin{enabstract}
  In subspace clustering, a group of data points belonging
  to a union of subspaces are assigned membership to their
  respective subspaces. This paper presents a novel method
  that regularizes sparse subspace representation by exploiting the
  structural sharing between tasks and data points via multitasking
  and group sparsity. By doing this, our algorithms are able to be more robust
  without introducing much additional computational cost. The
  theoretical analysis in this paper shows that under certain conditions
  exact clustering performance can be guaranteed. We demonstrate
  the advantage of the framework on Hopkins 155 dataset and NUS-WIDE dataset.

  \enkeywords{Subspace Clustering, Big Data, Unsupervised Learning, Group LASSO,
  Multitasking}{O24}
\end{enabstract}
