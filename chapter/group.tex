\section{G-SSC的理论分析}\label{sec:proof_group}
类似于MT-SSC,我们希望能得到G-SSC算法结果
自表示的充分条件,简单起见,本节不考虑噪音.

将\eqref{eq:group} 的限制条件去掉, 得到等价的优化问题
\begin{equation} \label{eq:group_nost}
  \min_{c} \|c\|_{\Omega,1}+\frac{\lambda}{2} \|x_i-X_{\{i\}\rightarrow
0}c\|_2^2,
\end{equation}
由于不考虑噪音,\eqref{eq:group_nost}可以进一步简化为
\begin{equation} \label{eq:group_nonoise}
    \min_{c} \|c\|_{\Omega,1} \quad\text{s.t.}\quad
    Y_{\{i\}\rightarrow 0} c=y_i
\end{equation}
我们使用$Y$和$y_i$表示无噪音的数据点.

为了得到\eqref{eq:group_nonoise}的自表示条件,先考虑一般的优化问题
\[
  \bP_2(b,A): \,\, \min_c\, \| c\|_{\Omega,1} \quad \text{s.t.}\quad Ac = b
\]
其对偶问题为 
\[
  \bD_2(b,A): \,\, \max_{v}\, b^T v \quad
  \text{s.t.}\quad \|A^Tv\|_{\Omega,\infty} \leq 1
\]
其中某个向量$x$ 的$\|\cdot\|_{\Omega, \infty}$的范数定义为
$\max_{\cI \in \Omega} \|x_\cI\|_2$.

类似于MT-SSC, 我们先对一般的优化问题 $\bP_2(b,A)$ 进行分析.
\begin{lemma}\label{lem:prim_dual}
  给定矩阵$A\in\R^{d\times n}$, 向量$b\in \R^d$, 指标集$\cT\subseteq [n]$以及
  指标集族$\Omega$满足\eqref{eq:nointer} 和\eqref{eq:full},
  且$\cT$是$\Omega$中某些指标集的并.
  若存在$c\in \R^n $使得$b = Ac$成立, $c$的支撑集$\cJ \subseteq \cT$,
  同时又存在$v\in\R^d$, 对$\forall \cI \in \Omega$有
  \begin{align}
    A^T_\cI v= \frac{c_\cI}{\|c_\cI\|_2} \quad& \text{if}\quad \cI \cap \cJ \neq
    \emptyset,\label{eq:group_c1}\\
    \|A^T_\cI v\|_2 \le  1 \quad &\text{if}\quad \cI \subseteq \cT,\label{eq:group_c2} \\
    \|A^T_\cI v\|_2< 1\quad &\text{if}\quad \cI \nsubseteq \cT\label{eq:group_c3}.
  \end{align}
  那么$\mathbf{P}_2(b,A)$的最优解$c^*$的支撑集$\cJ^*\subseteq \cT$.
\end{lemma}
\begin{proof}
  首先由于$\Omega$中的指标集两两交为空, 而$\cT$是$\Omega$中某些指标集的并, 
  所以对某个$\Omega$中的指标集$\cI$, 若$\cI \nsubseteq \cT$, 则
  $\cI \cap \cJ\subseteq \cI \cap \cT =\emptyset $, 因此\eqref{eq:group_c1}和\eqref{eq:group_c3}
  的不会矛盾.

  对于最优解$c^*$, 我们有
  \begin{align}
    \|c^*\|_{\Omega, 1} =& \sum_{\cI \cap \cJ\neq \emptyset}\|c^*_\cI\|_2 +
    \sum_{\cI \cap \cJ = \emptyset} \|c^*_\cI\|_2 \nonumber \\
    \ge& \sum_{\cI \cap \cJ\neq \emptyset}\|c_\cI\|_2 + 
    \sum_{\cI \cap \cJ\neq \emptyset}\langle \frac{c_\cI}{\|c_\cI\|_2} , c^*_\cI-c_\cI\rangle +
    \sum_{\cI \cap \cJ=\emptyset}\|c^*_\cI\|_2 \nonumber \\
    =& \sum_{\cI \cap \cJ\neq \emptyset}\|c_\cI\|_2 + 
    \sum_{\cI \cap \cJ\neq \emptyset}\langle v, A_\cI(c^*_\cI-c_\cI)\rangle +
    \sum_{\cI \cap \cJ =\emptyset}\|c^*_\cI\|_2 \nonumber \\
    =& \sum_{\cI \cap \cJ\neq \emptyset}\|c_\cI\|_2 +
    \sum_{\cI \cap \cJ = \emptyset} \left[\|c^*_\cI\|_2 -
    \langle v, A_\cI c^*_\cI\rangle\right],\label{eq:group_low1}
  \end{align}
  其中第一个不等号是因为$\|c^*_\cI\|_2 \ge \langle c_\cI/\|c_\cI\|_2,c^*_\cI\rangle$,
  第一个等号直接带入\eqref{eq:group_c1}可得,
  第二个等号是因为$c,c^*$满足$Ac=Ac^*=b$, 于是$\sum_{\cI\cap\cJ \neq
  \emptyset}A_\cI(c_\cI^*-c_\cI)=
  -\sum_{\cI \cap\cJ =\emptyset} A_\cI(c_\cI^*-c_\cI)$,
  而$c_{\cJ^c}=\mathbf{0}$, 所以带入可得. 由于
  $$\langle v, A_\cI c_\cI^*\rangle = \langle A_\cI^T v,
  c_\cI^*\rangle \le \|A_\cI^T v\|_2 \|c_\cI^*\|_2,$$
  所以\eqref{eq:group_low1}可以进一步放缩成
  \begin{align}
    \|c^*\|_{\Omega, 1} \ge& \sum_{\cI \cap \cJ\neq \emptyset}\|c_\cI\|_2 +
    \sum_{\cI \cap \cJ = \emptyset} \left(1-\|A_\cI^T v\|_2\right)\|c^*_\cI\|_2 \nonumber \\
    \ge& \|c\|_{\Omega, 1} + \sum_{\cI \nsubseteq \cT}\left(1-\|A_\cI^T
    v\|_2\right)\|c^*_\cI\|_2,
    \label{eq:group_low2}
  \end{align}
  其中用了\eqref{eq:group_c2}. 因为$c^*$是最优解,
  所以$\|c^*\|_{\Omega,1}\le \|c\|_{\Omega, 1}$,
  结合\eqref{eq:group_c3}可得$c^*_\cI = \mathbf{0},\, \cI \nsubseteq \cT$.
\end{proof}

\subsection{自表示条件}
设点$y_i\in\cS_\ell$, 为了构造满足满足\autoref{lem:prim_dual}条件的$c$和$v$,
我们先将优化问题的数据限制在子空间$\cS_\ell$中,
即先考虑$\bP_2(y_i, Y_{\cI_\ell^c\cup \{i\}\rightarrow 0})$.
$\bP_2(y_i, Y_{\cI_\ell^c\cup \{i\}\rightarrow 0})$
的最优解$c$满足$y_i=Y_{\{i\}\rightarrow 0}c$, 且支撑集$\cJ \subseteq \cI_\ell$,
而由强对偶性可得$\bD_2(y_i, Y_{\cI_\ell^c\cup \{i\}\rightarrow 0})$的最优解$v$,
对$ \cI\in \Omega$ 满足
\begin{align*}
  (Y_{\{i\}\rightarrow 0})^T_\cI v = \frac{c_\cI}{\|c_\cI\|_2}, \quad&
  \text{if}\quad\cI \cap \cJ \neq \emptyset \\
  \|(Y_{\{i\}\rightarrow 0})^T_\cI v\|_2 \le 1, \quad&
  \text{if} \quad \cI \subseteq \cI_\ell 
\end{align*}
因此如果
\begin{equation} \label{eq:group_cond1}
  \|Y_\cI^T v\|_2<1, \quad \text{if} \quad\cI \nsubseteq \cI_\ell
\end{equation}
成立, 那么根据\autoref{lem:prim_dual}, $\bP_2(y_i, Y_{\{i\}\rightarrow 0})$
的最优解$c^*$有$c^*_{\cI_\ell^c}=\mathbf{0}$, 即满足自表示性.

我们把\eqref{eq:group_cond1} 写成等价形式
\begin{equation*}
  \left\|Y_\cI^T \frac{v}{\|v\|_2}\right\|_2 \|v\|_2<1,
\end{equation*}
\(\|Y_\cI v/\|v\|_2\|_2\)由子空间之间的不相干性控制, 所以我们主要考察
$\|v\|_2$的上界.由于$v$是$\bD_2(y_i, Y_{\cI_\ell^c\cup \{i\}\rightarrow 0})$
的最优解,而$Y_{\cI_\ell^c\cup \{i\}\rightarrow 0}$的非零列和$y_i$都在子空间
$\cS_\ell$中, 所以$\bD_2(y_i, Y_{\cI_\ell^c\cup \{i\}\rightarrow 0})$ 一定存在
最优解属于$\cS_\ell$. 不妨设$v\in \cS_\ell$, 同时$v$属于集合
$$\{ v: \|Y^T_{\cI_\ell^c\cup \{i\}\rightarrow 0} v\|_{\Omega,\infty}\le
1\}.$$
为了给出$\|v\|_2$的上界, 我们需要\autoref{lem:group_rR}.
\begin{lemma}\label{lem:group_rR}
  给定矩阵$A\in \R^{d\times n}$, 指标集族$\Omega$和子空间$\cS\subset \R^d$,
  满足$A$中的每列$A_i\in \cS$, 定义集合
  \begin{align*}
    \cK &:= \left\{ Ab: \|b\|_{\Omega, 1}\le 1, b\in \R^n \right\},\\
    \cK^\circ &:= \left\{ u\in \cS : \|A^Tu\|_{\Omega, \infty}\le 1 \right\},
  \end{align*}
  则对 $u\in \cK^\circ$ 有
  $$ \|u\|_2 \le \frac{1}{r(\cK, \cS)}.$$
\end{lemma}
\begin{proof}
  $\cK$和$\cK^\circ$都是关于原点中心对称的凸包, 所以
  对任意$a\in \cS, \|a\|_2\le r(\cK, \cS)$, 有 $a \in \cK$,
  即存在$ b\in\R^n, \|b\|_{\Omega, 1}\le 1$, 使得$a=Ab$. 则$\forall u\in
  \cK^\circ$,
  有
  \begin{equation*}
    |a^T u| = |b^T A^T u| \le \sum_{\cI\in \Omega} \|b_\cI\|_2 \|A^T_\cI u\|_2
    \le \|b\|_{\Omega, 1}\|A^T u\|_{\Omega,\infty} \le 1.
  \end{equation*}
  由$a$的任意性可得
  $$ \|u\|_2 \le \frac{1}{r(\cK, \cS)}.$$
\end{proof}
根据\autoref{lem:group_rR},我们定义
\begin{equation*}
  \cQ_\Omega := \left\{ Y^T_{\cI_\ell^c\cup \{i\}\rightarrow 0}b:
  \|b\|_{\Omega, 1}\le 1, b\in \R^n \right\},
\end{equation*}
则 $\|v\|_2 \le 1/r(\cQ_\Omega, \cS_\ell)$.
因此,\eqref{eq:group_nonoise}的解满足自表示性的充分条件为
\begin{equation} \label{eq:group_cond2}
  \left\|Y_\cI^T \frac{v}{\|v\|_2}\right\|_2 < r(\cQ_\Omega, \cS_\ell), \quad
  \cI \nsubseteq \cI_\ell, \forall\, \cI \in \Omega.
\end{equation}
如果我们把\eqref{eq:group_cond2}和\cite[定理~2.5]{soltanolkotabi2012geometric}
的条件$|\langle y_i, v/\|v\|_2\rangle|<r(\cQ^\ell_{-i}, \cS_\ell)$
相比较可以发现,\eqref{eq:group_cond2}的不等式左右两项都较大, 所以理论上很难
判断那种算法更好.
