\chapter{多任务和组稀疏方法}\label{chp:prob_setup}
虽然SSC是目前最优秀的子空间聚类算法之一,但是它有两个缺陷:第
一,SSC 每次只考虑一个点相对于其它点的表示,
这比较容易受到噪音和异常点的干扰, 例如考虑一种低端情况:
有三个来自不同子空间的点在一条直线上, 那么它们很可能会互相稀疏表示,
这样就偏离了稀疏表示的期望.
如果我们预先知道点\(x_1, x_2, x_3\)
属于同一类,然后寻找其它能同时表示\(x_1, x_2, x_3\)的点, 这无疑比
单独考虑\(x_1\)或\(x_2\)或\(x_3\)更稳健. 第二, 在SSC 模型中,
我们要求表示尽量稀疏, 然而过分稀疏的邻接矩阵, 可能在
接下来的谱聚类中没有比较好的可分性. 事实上最优的邻接图应该是
同一子空间上的点之间连接尽量多,不同子空间的点之间连接尽量少.
如果我们能让表示向量既不对应不同子空间的点又不集中在
少数几个点上,这样产生的邻接图聚类效果会更好.

基于上面的考虑, 本文结合压缩感知中的联合稀疏(Joint sparse) 模型~\cite{yuan2012visual},
提出了新的子空间聚类方法, 它的核心内容分两步:
第一步是根据点的空间信息, 先将某些点聚在一起, 组成若干组.
第二步是求解自表示系数, 有两种方法:
(1) 将一组点作为整体相对其它点做回归,
要求他们被同时表示, 这是\emph{多任务稀疏子空间聚类}
(Multi-task sparse subspace clustering, MT-SSC);
(2) 将一个点用其它点表示, 但是采用组稀疏的正则化范数,
即\emph{组稀疏子空间聚类}(Group sparse subspace clustering, G-SSC).
两种方法都可以得到表示系数, 进而构造邻接矩阵, 最后谱聚类.

\section{分组算法} 
第一步要将数据点分成若干组, 尽量使每一组里的点属于同一个子空间.
为了做到这点最直接的想法是用\(K\)近邻算法, 将每个点和它的邻居分为一组.
然而如果只考虑数据点之间的欧式距离, 无疑忽略了线性空间的性质.
所以我们类似于~\cite{heckel2013noisy}, 用两点之间的夹角作为距离度量,即
\[d(x_i,x_j)=\exp\left\{-\cos^{-1}\left( \frac{|\langle x_i,
x_j\rangle|}{\|x_i\|_2\|x_j\|_2} \right)\right\}.\]
不过如果单纯考虑夹角,也只能利用一个维度上的信息, 我们更近一步地定义
点到点集的投影距离
\[ d(a, A, r) =  \|a\|_2-\|U^Ta\|_2 \]
其中\(a \in R^d\)代表一个数据点,\(A\in \R^{d\times m}\)是一组数据点组成的矩阵,
\(r\)为正整数, \(U\)是对\(A\)做奇异值分解后的前\(\min(r, \rank(A))\)个左奇异向量组成的矩阵.
\(r\)的大小可以
事先给定,也可以取\(A\)的有效秩~\cite{yan2006general}:
\[
  r = \argmin_{\tldr} \frac{\sigma_{\tldr+1}^2}{\sum_{i=1}^{\tldr}
  \sigma_{\tldr}^2} +\kappa \tldr,
\]
其中\(\sigma_1,\sigma_2,\cdots,\sigma_m\)是\(A\)的奇异值,\(\kappa\)是和噪音大小有关的参数.

我们从一个点\(x_0\)出发,  自然取它的\(k\)个夹角最近邻
\(x_1, x_2, \ldots, x_k \)为一组,于是邻居集合
\(\cN := \left\{ x_0, x_1, \ldots, x_k \right\}.\)
接着用这\(k+1\)个点张成\(r\)维平面, 我们得到投影矩阵\(U\),再取\(k\)个投影最近邻,更新\(\cN\)和
\(U\),如此循环, 退出条件是最大相对误差大于阈值\(\theta\),即
\[\max_{x\in \cN} \frac{\|x\|_2-\|U^T x\|_2}{\|x\|_2} > \theta.\]
于是我们得到计算每个点的拓展最近邻的\autoref{alg:nsn}. 
\begin{algorithm}[tb] \caption{拓展最近邻}\label{alg:nsn}
  \begin{algorithmic}
    \State \textbf{输入:} \(n\) 个数据点 \(\cX = \{x_1,\ldots,x_n\}\),
    \(\cX\)中的某个数据点\(x_0\), 邻居增长数\(k\),子空间维度\(r\)(或有效秩参数\(\kappa\)),噪音控制参数\(\theta\).
    \State 初始化:将所有向量归一化,\(U = x_0, \cX = \cX \setminus x_0,
    \cN=\{x_0\}\) 
    \Repeat
    \State{1. } 对每个\(x\in \cX\)计算\(\|U^Tx\|_2\),取出前\(k\)大的点组成集合\(\cN_0\)
    \State{2. } \(\cX = \cX \setminus \cN_0, \cN = \cN \cup \cN_0\)
    \State{3. } \(U\)取为\(\cN\)中所有点组成的矩阵做奇异值分解的前\(r\)个左奇异向量
    (或取\(r\)为\(\cN\)中点组成矩阵的有效秩)
    \State{4. } \(\text{err}=1 - \min_{x \in \cN}\|U^T x\|_2\)
    \Until{\(\text{err}>\theta\)}
    \State \textbf{输出:} \(x_0\)的拓展最邻居集合\(\cN \setminus \cN_0\)
    \Comment{有可能只有\(x_0\)自身}
  \end{algorithmic}
\end{algorithm}

由于我们引入了阈值\(\theta\), 使得\autoref{alg:nsn} 有一定的自我修正能力.
如果邻居增长数\(k\)取得过大, 或者循环进行多步后, 很可能会在邻居集合\(\cN\)
中混入其它子空间的点, 这时\(\cN\)中的点就不能很好地分布在一个\(r\)维平面上,
使得\(\text{err}>\theta\), 然后我们会从\(\cN\)中删去所有上一步加入的点, 这样
混入的错误点又会被剔除. 

得到每个点的近邻之后, 还要把点分成若干组, 因为这些近邻之间有重叠,
但是后面的MT-SSC和G-SSC要求每个点只能在一个组中.
当然这里可以使用谱聚类等聚类算法, 不过简单起见, 我们直接采用``先到先得''的原则:
先找到一个未分组的点\(x_0\). 将其近邻集合\(\cN(x_0)\)中没有分组的和它分成一组,
再找下一个未分组点, 直到分完所有点.
注意每个点都一定和它邻居集合的子集在一组, 所以如果邻居集合没有包含错误点,
那么分组也一定没有错误, 如果邻居集合包含错误点, 也可能会得到正确的分组,
因为邻居集合有重叠.
实际计算中,为了消除取点带来的随机性,我们总是按照点的\(\ell_2\)范数从大到小,依次选取.
对于每个分组, 我们将其所含点的编号组成一个指标集\(\cI \subset [n]\),
于是分组算法最终可以给出一个指标集族\(\Omega\), 满足
\begin{gather}
  \cI \cap \cJ = \emptyset,\quad \forall \,\cI, \cJ \in \Omega\quad
  \cI \neq \cJ, \label{eq:nointer}\\
  \bigcup_{\cI\in \Omega} \cI = [n].\label{eq:full} 
\end{gather}
这保证了\(\Omega\)是对所有数据点的一个剖分, 从而有后面的多任务和组稀疏方法.

\section{多任务和组稀疏方法}
先介绍一些记号: 我们用\(X_i\) 表示矩阵\(X\) 的第\(i\)列,
\(X_{(i)}\)表示矩阵\(X\)的第\(i\)行,
\(X_{i,j}\)表示矩阵的第\(i\)行第\(j\)列的元素.
对某个指标集\(\cI\subset [n]\), \(\cI^c:=[n]\setminus \cI\).
\(x_\cI\)表示向量\(x\)中对应指标集\(\cI\)的部分,
\(X_\cI\)表示将矩阵\(X\)的列向量按照指标集\(\cI\)取出后组成的子矩阵, 
\(X_{(\cI)}\)表示将矩阵\(X\)的行向量按照指标集\(\cI\)取出后组成的子矩阵,
\(X_{\cI, \cJ}\)表示行按指标集\(\cI\)抽出, 列按指标集\(\cJ\)抽出组成的子矩阵. 
给定满足\eqref{eq:nointer} \eqref{eq:full} 的指标集族\(\Omega\),
定义向量范数\(\|x\|_{\Omega, 1}:=\sum_{\cI \in \Omega} \|x_{\cI}\|_2\),
矩阵范数\(\|X\|_{\Omega, 1}:=\sum_{i} \|X_i\|_{\Omega, 1}\).
再定义矩阵范数\(\|X\|_{2, 1}:=\sum_{i} \|X_i\|_2\).

\subsection{MT-SSC}

SSC对每个点\(x_i\)求解\eqref{eq:ssc_noise}, 
而有了指标集族\(\Omega\)后, 对于\(\cI\in \Omega\),
可以得到多列数据组成的矩阵\(X_\cI\), 我们希望找到一些点能同时表示
\(X_\cI\)的每一列, 即考虑下面的优化问题
\begin{equation}\label{eq:multi}
  \begin{gathered}
    \min_{C}\; \|C^T\|_{2, 1} + \frac{\lambda}{2} \|X_\cI - XC\|_F^2\\
    \text{s.t.} \quad C_{(\cI)} = \mathbf{0},
  \end{gathered}
\end{equation}
\(C \in \R^{n \times |\cI|}\)是点集\(\{x_i, \forall\, i\in \cI\}\)的表示矩阵,
\(\ell_{2,1}\)范数的正则化项能使 \(C\)在非零行向量较少, 而非零行向量对应的点
能同时表示\(X_\cI\)的每一列, 约束条件是为了排除自己表示自己这种平凡情形.
如果我们所有分组一起考虑, 把\eqref{eq:multi} 改写成
\begin{equation}\label{eq:multi_matrix}
  \begin{gathered}
    \min_C\; \|C^T\|_{\Omega, 1} + \frac{\lambda}{2} \|X-XC\|_F^2 \\
    \text{s.t.} \quad C_{\cI, \cI} = \mathbf{0}, \quad \forall \, \cI \in
    \Omega,
  \end{gathered}
\end{equation}
其中\(\|\cdot\|_{\Omega, 1}\)为矩阵每列的\(\ell_{\Omega, 1}\)范数加和.
我们要用\eqref{eq:multi_matrix} 的解\(C\)组成邻接矩阵\((|C|+|C^T|)/2\),
进而做谱聚类. 不过不同于SSC, 我们在每个分组内部缺少邻接关系,
因此对\(i, j\in\cI\) 取\(C_{i, j} = \max(|C_\cI|)\), 这样就能构造出完整的邻接矩阵.
MT-SSC同时考虑多个点的回归, 所以应该会比SSC更加稳健, 同时注意到\eqref{eq:multi} 
的解在非零行上应该是稠密的, 则其生成的邻接图在同一子空间内部的连接会更紧密.

\subsection{G-SSC}
不同于\eqref{eq:ssc_noise},  G-SSC对每个\(x_i\)考虑下面的优化问题
\begin{equation}
  \begin{gathered}
    \min_{c} \; \|c\|_{\Omega, 1} + \frac{\lambda}{2} \|x_i - Xc\|_2^2\\
    \text{s.t.} \quad [c]_i = 0.
  \end{gathered}
  \label{eq:group}
\end{equation}
其中\(\cI\)是\(\Omega\)中包含\(i\)的指标集, 由\eqref{eq:nointer} 和\eqref{eq:full} 
可知\(\cI\)存在且唯一.\eqref{eq:group} 相当于用某些组的点来表示\(x_i\), 相比于
独立考虑每个点, 这样能减少异常点的干扰, 也能使得到的表示矩阵相对稠密.
同样地,写出\eqref{eq:group} 的矩阵形式
\begin{equation}
  \begin{gathered}
    \min_C\; \|C\|_{\Omega, 1} + \frac{\lambda}{2} \|X-XC\|_F^2 \\
    \text{s.t.} \quad \diag(C) = \mathbf{0}.
  \end{gathered}
  \label{eq:group_matrix}
\end{equation}
用\eqref{eq:group_matrix} 的解\(C\)组成邻接矩阵\((|C|+|C^T|)/2\), 进而做谱聚类.

注意如果\autoref{alg:nsn} 的参数\(\theta\)取得很小, 那么每个点的最近邻集合
将只包含自己, 于是MT-SSC和G-SSC将退化成SSC, 所以如果参数选取适当, MT-SSC
和G-SSC的聚类结果不会比SSC差.

\subsection{ADMM算法}
下面讨论如何求解MT-SSC和G-SSC的优化问题.
\eqref{eq:multi_matrix} 和\eqref{eq:group_matrix} 可以写成统一的形式
\begin{equation}\label{eq:MatrixLasso}
  \begin{gathered}
    \min_{C} \; \|C\|_{\Omega}+\frac{\lambda}{2}\|X-XC\|_F^2  \\
    \text{s.t.} \quad \Bdiag(C)=\mathbf{0},
  \end{gathered}
\end{equation}
其中\(\|\cdot\|_{\Omega}\) 为某个与指标集族\(\Omega\)有关的凸范数,
对于MT-SSC, \(\|C\|_{\Omega}=\|C^T\|_{\Omega, 1}\), 
\[
  \Bdiag(C)=\begin{bmatrix}
    C_{\cJ_1, \cJ_1}  & \mathbf{0} & \cdots & \mathbf{0} \\
    \mathbf{0}  & C_{\cJ_2, \cJ_2} & \cdots & \mathbf{0} \\
    \vdots   &    \vdots    &    \ddots     & \vdots \\
    \mathbf{0}  &  \mathbf{0} &  \cdots  & C_{\cJ_{|\Omega|},\cJ_{|\Omega|}}
  \end{bmatrix}
\]
其中\(\{\cJ_1, \cJ_2,\ldots,\cJ_{|\Omega|}\}=\Omega\),
对于G-SSC, \(\|C\|_{\Omega}=\|C\|_{\Omega, 1}, \Bdiag(C) = \diag(C)\).

我们用ADMM算法~\cite{stephen2011distributed} 求解\eqref{eq:MatrixLasso} 
如果适当交换\(X\)列向量的顺序, 将同一组的点排在一起, 
那么\eqref{eq:MatrixLasso} 可以写成下面的等价形式
\begin{equation}\label{eq:MatrixLasso_modify}
  \begin{gathered}
    \min_{C} \; \|C\|_{\Omega}+\frac{\lambda}{2}\|X-XJ\|_F^2 \\
    \text{s.t.} \quad J=C-\Bdiag(C).
  \end{gathered}
\end{equation}
我们写出\eqref{eq:MatrixLasso_modify} 的Lagrangian,
并在其上增加一个等价的二次惩罚项得到增广Lagrangian
\begin{multline*}
  \cL= \|C\|_{\Omega}+\frac{\lambda}{2}\|X-XJ\|_F^2 
  +\frac{\mu}{2}\|J-C+\Bdiag(C)\|_F^2\\
  +tr(\Lambda^T(J-C+\Bdiag(C))),
\end{multline*}
其中\(\Lambda\)是对偶变量, \(\mu\)是给定参数. 为了交替优化
\(J\), \(C\) 和 \(\Lambda\) 直到收敛, 需要求解 \(\partial \cL/\partial J=0\)
和 \(\partial \cL/\partial C=0\). 由\(\partial \cL/\partial J=0\) 可得
\[
  J=(\lambda X^TX+\mu I)^{-1}\left(\lambda X^TX+\mu C - 
  \mu\Bdiag(C)-\Lambda\right).
\]
另一方面, 虽然\(\cL\)关于\(C\)不可导, 但极值可以用拓展的
Soft-Threshold 算子~\cite{donoho1995noising}求得.
先定义
\begin{equation*}
  S_\alpha(a) = \begin{cases}
    \mathbf{0} & \|a\|_2 \le \alpha, \\
    (\|a\|_2 - \alpha) \frac{a}{\|a\|_2} & \|a\|_2 > \alpha.
  \end{cases}
\end{equation*}
其中\(\alpha\)为给定正数,\(a\)为任意向量.
再定义\(\sfth_{\alpha,\Omega}:\R^n\rightarrow \R^n\), 若
\(b =\sfth_{\alpha,\Omega}(a)\), 则\( b_\cI= S_\alpha(a_\cI),
\, \forall \cI \in \Omega\).

\begin{algorithm}[tb]
  \caption{求解MT-SSC和G-SSC的优化问题}
  \label{alg:MatrixSSC}
  \begin{algorithmic}
    \State {\bfseries 输入:}
    \(n\) 个样本点 \(x_1,\ldots,x_n\),它们组成了数据矩阵\(X\in \R^{d\times
    n}\),组别信息\(\Omega\),参数\(\lambda\) 和 \(\mu\)
    \State 初始化 \(C=0\), \(J=0\), \(\Lambda=0\)
    \While{未收敛}
    \State{1. } 更新 \(J\) 
    \[J=(\lambda X^TX+\mu I)^{-1}(\lambda X^TX+\mu C-\Lambda).\]
    \State{2. } 更新 \(C\)
    \[ C^{'}=\sfth_{\frac{1}{\mu},\Omega}\left(J+\Lambda/\mu\right), \]
    \[ C=C^{'}-\Bdiag(C^{'}).\]
    \State{3. } 更新 \(\Lambda\)
    \[\Lambda=\Lambda+\mu(J-C)\]
    \EndWhile
    \State 对MT-SSC要取每组的绝对值最大值将\(C\)补充完整
    \State {\bfseries 输出:} 最优解\(C\)
  \end{algorithmic}
\end{algorithm}
求解\eqref{eq:multi_matrix} 时, 我们将\(\sfth\)作用在矩阵
\(J+\Lambda/\mu\)的每一个行向量上,
求解\eqref{eq:group_matrix} 时作用于列向量,
最终得到\autoref{alg:MatrixSSC}.
注意到 \((\lambda X^TX+\mu I)^{-1}\)
能被预先算好,这样每次迭代的时间就主要花在计算矩阵乘法上,
因此MT-SSC,G-SSC花在优化上的时间和SSC几乎相同.
